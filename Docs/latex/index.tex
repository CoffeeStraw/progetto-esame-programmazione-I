\hypertarget{index_intro_sec}{}\section{Introduzione}\label{index_intro_sec}
Il progetto {\bfseries Gieson\+Fault} è stato realizzato come progetto per l\textquotesingle{}esame di Programmazione I al dipartimento di Matematica e Informatica di Perugia.~\newline
 Il progetto è stato creato con la forte attenzione alla produzione di un codice solido e tecnico. Una forte {\bfseries leggibilità del codice} e {\bfseries un\textquotesingle{}esperienza utente ottimale} sono state le priorità in fase di sviluppo.\hypertarget{index_run_sec}{}\section{Compilazione ed avvio}\label{index_run_sec}
\hypertarget{index_compilation}{}\subsection{Compilazione\+:}\label{index_compilation}

\begin{DoxyCode}
gcc -o Output main.c gamelib.c -Wall -std=c11
\end{DoxyCode}


Per entrare in modalità debug, (dove in partenza si avranno 99 oggetti di ogni tipo) è possibile compilare nel seguente modo\+: 
\begin{DoxyCode}
gcc -D DEBUG -o Output main.c gamelib.c -Wall -std=c11
\end{DoxyCode}
\hypertarget{index_run}{}\subsection{Avvio\+:}\label{index_run}

\begin{DoxyCode}
./Output
\end{DoxyCode}
 